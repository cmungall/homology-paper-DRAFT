\section{Computational Anatomical
Homology}\label{computational-anatomical-homology}

\section{Abstract}\label{abstract}

\section{Background}\label{background}

Homology was originally defined as ``the same organ in different animals
under every variety of form and function''\cite{Owen1843}. With the
advent of molecular biology, the concept of homology came to cover the
sequences of genes and proteins. Fitch introduced subtypes such as
paralogy and orthology, and provided rules for calculating membership in
these categories based relationships on gene and protein
trees\cite{Fitch1970}. This formulation forms a crucial bedrock in the
field of bioinformatics. In contrast, the application of the homology
concept in its original usage in the study of anatomy has retained more
of a philosophical flavor.

Here we provide a formulation of anatomical homology that can be applied
in modern bioinformatics information systems, used to answer questions
about the relationships between genes, anatomical structures and their
histories. Our formulation is expressed using the Web Ontology Language
(OWL), so we first provide prelininary information on this language.

Our effort builds on and contrasts with previous efforts, such as the
Vertebrate Bridging Ontology (VBO) which catalogued a set of anatomical
structures from a number of vertebrates, together with homology
relationships connecting them. We also build on our previous efforts, on
constructing resources such as the Vertebrate Homologous Organ Group
(VHOG) ontology\cite{Bgee} and the Vertebrate Skeletal Anatomy Ontology
(VSAO)\cite{Dahdul}, both of which have been merged into the Uberon
metazoan anatomy ontology\cite{Mungall2012},\cite{Haendel2014}.

\subsection{Ontologies and Computational
Logic}\label{ontologies-and-computational-logic}

Ontologies are symbolic representations of some aspect of the world that
can be used by both humans and by machines. Ontologies are commonly
conceived of as semantic networks\cite{}, with nodes representing terms
from the domain of discourse (for example, ``limb'', ``bone'', ``skull''
and so in anatomy) and the edges forming relationships that hold between
the correspoding entities (for example, femur partOf hindlimb).

There are a variety of languages used for expressing an ontology model,
and these languages fall into different families, each with
characteristics that have impacts on how the ontology is developed and
applied. Three languages of note are Resource Description Framework
(RDF), the Web Ontology Language (OWL) and Common Logic (CL). Of these
3, CL is the most expressive, covering all of First Order Logic (FOL)
(with some aspects of second-order). Any RDF or OWL ontology can be
represented in CL. Formally, OWL is a more expressive subset of RDF
(every OWL ontology is an RDF ontology, but OWL defines constructs with
greater expressive power).

The language that is of most relevance here is OWL, as this is what is
currently used for the majority of ontologies in the field of biology
(many biological ontologies are authored using a format called OBO,
which is officially an OWL syntax). One of the features of OWL that is
most salient to the problem of homology is its strict partitioning of
elements into disjoint categories, the three core ones being
\emph{classes}, \emph{individuals} and
\emph{relations}\footnote{in OWL terminology,
an Object Property}.

When settling down to model some aspect of the world in OWL, it is
necessary for the modeler to decide what to put in each of these boxes.
The choice may be dictated by philosophical consideration, but pragmatic
aspects dictated by mathematical-logical properties of these sets
usually dominate. The most common approach (although by no means
universal) is model ``particulars'' as individuals - this may include
your friend \texttt{Joe} or his dog \texttt{Fido}. Joe may be related to
Fido via the \texttt{owns} relation. Classes are essentially sets of
individuals, so we may choose to include classes for \texttt{human} and
for \texttt{dog}, with Joe and Fido instantiating these classes
respectively. It is valid, if unusual, to model species or taxa as
individuals. Note there is a longstanding question in the philosophy of
biology concerning whether species are classes or individuals.

Crucially for our purposes here, the only relationships that can hold
between classes are set-theoretic ones. For example, \texttt{human} and
\texttt{dog} may both be subclasses of the class \texttt{mammal}
(i.e.~the set of humans is a subset of the set of mammals).
\texttt{human} and \texttt{dog} may stand in a disjointness relationship
(i.e.~their set intersection is empty).

A vertebrate anatomical ontology will typically include classes such as
\texttt{digit}, \texttt{phalanx}, as well as more specific classes such
as \texttt{phalanx of hand} or even
\texttt{distal phalanx of digit 2 of left hand}. Within this model,
these classes instantiated by particulars in the world; for example,
Joe's left thumb, which instantiates the class \texttt{manual digit 1}
as well as more general classes such as \texttt{digit} and more specific
classes such as \texttt{human left manual digit 1}.

Ontologies are chiefly concerned with generalizations, such as the fact
that digits are always part of the autopod (hand or foot). This can be
done using the mereological partOf relation, but in OWL relations always
hold between individuals, such as Joe's left manual digit 1 and Joe's
left hand. There is no physical, mereological relation between the
classes, which are abstractions over the individual elements. The actual
relationship between the classes is a set-theoretic one, which can be
stated in normal set-theoretic notation as:

\begin{verbatim}
digit ⊆ { x : ∃ y : y ∈ autopod & partOf(x,y) }
\end{verbatim}

i.e.~the \texttt{digit} set is a subset of the set of all things whose
members $x$ stand in a partOf relaitonship to some $y$ where y is an
autopod.

OWL is at its core a notation for these kinds of set-builder
expressions. The above set notation can be written in the Manchester
Syntax\cite{Horridge} variant of OWL as:

\begin{verbatim}
Class: digit SubClassOf: partOf some autopod
\end{verbatim}

A more compact (if abstruse) way of writing the same thing is in DL
notation:

\begin{verbatim}
digit ⊆  ∃partOf.autopod
\end{verbatim}

An understanding of the distinction between statements about classes
versus statements about individuals in OWL is of paramount importance to
understanding the implications of modeling a homologousTo relationship,
at the core of this paper.

\subsection{Existing work}\label{existing-work}

\begin{itemize}
\itemsep1pt\parskip0pt\parsep0pt
\item
  VBO
\item
  HOM
\item
  UBERON
\item
  Franz et al
\end{itemize}

\subsubsection{Uberon}\label{uberon}

Uberon contains classes for anatomical structures found in a variety of
animals, with an emphasis towards vertebrates. Classes in Uberon
represent groupings of structures; for example, the class \texttt{limb}
groups all tetrapod limbs, and the class \texttt{mammary gland} groups
all mammary glands (which are exclusive to mammals).

In the past we have characterized Uberon as ``homology neutral''. By
this we mean that there is no \emph{guarantee} that the existence of a
class in Uberon means that the class represents a group of homologous
structures; it does not mean that homology is explicitly avoided as a
grouping criteria. Homology is handled separately, in datafiles provided
by groups such as Phenoscape and BgeeDb. In practice, a strict
separation is difficult, and often it is impossible to avoid homology -
for example, the definition of ``manual digit 3'' is essentially
homological.

\section{Results}\label{results}

\subsection{Competency Questions and Evaluation
Strategy}\label{competency-questions-and-evaluation-strategy}

Here we elcudiate a number of different homology models and evaluate
their characteristics. We characterize two broad categories of
competency query for the purposes of evaluation.

We call the first category \texttt{anatomy queries} are queries whose
answers are anatomical structures; examples are:

\begin{itemize}
\itemsep1pt\parskip0pt\parsep0pt
\item
  what, if any, are the homologs of tetrapod digits in fish?
\item
  what is the evolutionary ancestor of tetrapod limbs and
  actinopterygian fins?
\item
  what structures are in the ontogenic lineage of both limbs and fins?
\end{itemize}

The second category are \texttt{data queries}, as they can be
characterized by the fact that anatomical relationships are used as a
means of enhancing a query about some data that has been indexed using
the ontology. We use genes as our paradigmatic data entities. Examples:

\begin{itemize}
\itemsep1pt\parskip0pt\parsep0pt
\item
  what vertebrate genes are expressed in either the human forelimb, or
  any homolog thereof?
\item
  to what extent are genes expressed in two homologous structures
  themselves homologs?
\item
  what zebrafish genes have similar anatomical phenotypes to human
  Hoxd13?
\end{itemize}

\subsection{Basic Homology Relation
Properties}\label{basic-homology-relation-properties}

Our approach is not to take the homology relation as primitive, but
instead to define it in terms of descent. This approach is compatible
with a broad range of existing treatments, and is in particular highly
compatible with the Fitchean formulation of sequence homology.

We first introduce a relation \texttt{descendedFrom}, which captures the
phylogenetic lineage between an entity and its antecedent. At this stage
we leave open the question of what kind of thing this relation connects,
i.e.~we do not yet commit to an information flow model\cite{Van Valen}
vs connecting physical structures.

This relation has the following characteristics:

\begin{itemize}
\itemsep1pt\parskip0pt\parsep0pt
\item
  Transitivity
\item
  Reflexivity
\item
  Anti-symmetry
\end{itemize}

The reflexivity characteristic means everything is descendedFrom itself;
this makes aspects of the formulation easier, and we can follow
analogous practice in mereology and introduce an irreflexive
\texttt{properlyDescendedFrom} if required. Similary for transitivity;
we can later introduce a \texttt{directlyDescendedFrom} (for example,
between a child structure and the antecedent structure in the direct
biological parent).

We use this relation to partially define \texttt{homologousTo}, using an
OWL property chain axiom:

\begin{verbatim}
homologousTo <- descendedFrom o inverse(descendedFrom)
\end{verbatim}

Which is shorthand for the following rule:

\begin{verbatim}
IF x descendedFrom a AND
   y descendedFrom a
THEN x homologousTo y
\end{verbatim}

It follows from this definition that homology is itself transitive and
reflexive. It also follows that homology is symmetric (see supplemental
proofs).

We assume this part of the formulation is rigid. From here there are
certain decision points, specifically with respect to what kinds of
things can be related using these relations. There are also further
distinctions to be made between interspecies (phyletic) homology and
serial homology (or between parology and orthology for sequences); we
will return to these distinctions later.

TODO: homologousAs

\subsection{Model Decision Tree}\label{model-decision-tree}

The first modeling question that needs to be answered is whether
generalized anatomical structures such as \texttt{digit} or
\texttt{hand digit 1} are to be modeled as classes or individuals. If
the answer is ``classes'', then the model is a \emph{TBox} model, and if
the answer is ``individuals'' then the model is an \emph{ABox} model.

As mentioned previously, the conventional approach is to admit only
spatiotemporal particulars such as ``Joe's left thumb'' to be
individuals, with all generalizations being classes. We will therefore
start with homology models that conform to this practice. Subsequently
we will return to ABox models.

\subsubsection{TBox Model 1: REAHM}\label{tbox-model-1-reahm}

Given that both \texttt{pectoral fin} and \texttt{forelimb} are modeled
as classes, we cannot make a direct logical homology relationship
between these two, as they are not individuals in the domain of
discourse. Such a statement would be syntactically invalid OWL.

Instead, we do what is standard for ontologies are introduce an
existential restriction axiom (see previous example for partOf). For
example, it is syntactically valid to state:

\begin{verbatim}
Class: 'pectoral fin' SubClassOf: homologousTo some 'forelimb'
\end{verbatim}

Which is shorthand for ``every pectoral fin is homologous to some
forelimb''

\begin{verbatim}
pectoral-fin ⊆ { x : ∃ y : y ∈ forelimb & homologousTo(x,y) }
\end{verbatim}

One crucial point of note here is that the above assertion does
\emph{not} entail the reciprocal assertion, namely that every forelimb
is homologous to some pectoral fin. This is despite the fact that the
homology relation has the property of symmetry (symmetry only holds at
the level of individuals).

Therefore, as a matter of practice, each assertion should be accompanied
by a reciprocal assertion. This rule cannot be stated in OWL (but it
would be possible to have some kimd of meta-reasoning system on top,
rather generates these). For example:

\begin{verbatim}
Class: 'pectoral fin' SubClassOf: homologousTo some 'forelimb'
Class: 'forelimb' SubClassOf: homologousTo some 'pectoral fin'
\end{verbatim}

This practice lends itself to the name Reciprocal Existential Axiom
Homology Model (REAHM).

This model yields inferences that are sometimes too conservative
(incomplete) and sometimes too liberal (too inclusive).

As an example of the incompleteness problem, consider the ontology
above, and the question: ``is a zebrafish pectoral fin homologous to a
human limb?''. More formally, is the following OWL axiom entailed?

\begin{verbatim}
('pectoral fin' and partOf some 'Danio rerio')
   SubClassOf homologousTo some
('forelimb' and partOf some 'Homo sapiens') 
\end{verbatim}

The axiom would not be entailed under the above ontology, yet it is
valid.

As an example of inclusivity, consider the query ``what is homologous to
some fin'' on the following ontology:

\begin{verbatim}
Class: 'pectoral fin' SubClassOf: homologousTo some 'forelimb'
Class: 'forelimb' SubClassOf: homologousTo some 'pectoral fin'
Class: 'pectoral fin' SubClassOf 'paired fin'
Class: 'pelvic fin' SubClassOf 'paired fin'
Class: 'forelimb' SubClassOf 'limb'
Class: 'hindlimb' SubClassOf 'limb'
\end{verbatim}

The entailed answers would be:

\begin{itemize}
\itemsep1pt\parskip0pt\parsep0pt
\item
  limb
\item
  forelimb
\item
  hindlimb
\end{itemize}

The answers include classes at a `different level' from limb; whether
this is deemed too inclusive depends on the model in question.

\subsubsection{TBox Model 2: D-REAHM}\label{tbox-model-2-d-reahm}

A slight variant of this model is to make assertions only using the
decendedFrom relation, and to infer homologousTo. For example:

\begin{verbatim}
Class: 'pectoral fin' SubClassOf: decendedFrom some 'pectoral-fin forelimb ancestor'
Class: 'forelimb' SubClassOf: decendedFrom some 'pectoral-fin forelimb ancestor'
\end{verbatim}

We call this variant D-REAHM (with ``D'' standing for ``Decendant'').
{[}call this DAHM?{]}

We might naively expect that we could use these assertions to infer the
horizontal homologousTo axioms, but this is not the case; we would need
to add reciprocal axioms between ancestor and descendant in order to
infer this. (see supplemental proofs).

\subsubsection{TBox Model 3: VAHM}\label{tbox-model-3-vahm}

This modifies the D-REAM model to use an individual as the ancestor

\begin{verbatim}
Class: 'pectoral fin' SubClassOf: decendedFrom value 'pectoral-fin forelimb ancestor'
Class: 'forelimb' SubClassOf: decendedFrom value 'pectoral-fin forelimb ancestor'
\end{verbatim}

In contrast to DREAM, this states that the pectoral fin and forelimb
share the \emph{same} ancestor.

This model gives \emph{highly promiscuous} entailments. For example,
given the ontology \texttt{minimal appendage}:

\begin{verbatim}
Class: 'paired fin' SubClassOf: decendedFrom value 'paired-fin/limb ancestor'
Class: 'limb' SubClassOf: decendedFrom value 'paired-fin/limb ancestor'
Class: 'pectoral fin' SubClassOf 'paired fin'
Class: 'pelvic fin' SubClassOf 'paired fin'
Class: 'forelimb' SubClassOf 'limb'
Class: 'hindlimb' SubClassOf 'limb'
\end{verbatim}

If we are to ask the question ``what is homologous to the `pectoral
fin'\,'', the answers would be:

\begin{itemize}
\itemsep1pt\parskip0pt\parsep0pt
\item
  `pelvic fin'
\item
  `forelimb'
\item
  `hindlimb'
\end{itemize}

This is clearly more answers than is desired.

\subsubsection{ABox Models}\label{abox-models}

\subsection{}\label{section}

\section{Discussion}\label{discussion}

\section{Conclusions}\label{conclusions}

\section{References}\label{references}
